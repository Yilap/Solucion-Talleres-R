% Options for packages loaded elsewhere
\PassOptionsToPackage{unicode}{hyperref}
\PassOptionsToPackage{hyphens}{url}
%
\documentclass[
]{article}
\usepackage{amsmath,amssymb}
\usepackage{iftex}
\ifPDFTeX
  \usepackage[T1]{fontenc}
  \usepackage[utf8]{inputenc}
  \usepackage{textcomp} % provide euro and other symbols
\else % if luatex or xetex
  \usepackage{unicode-math} % this also loads fontspec
  \defaultfontfeatures{Scale=MatchLowercase}
  \defaultfontfeatures[\rmfamily]{Ligatures=TeX,Scale=1}
\fi
\usepackage{lmodern}
\ifPDFTeX\else
  % xetex/luatex font selection
\fi
% Use upquote if available, for straight quotes in verbatim environments
\IfFileExists{upquote.sty}{\usepackage{upquote}}{}
\IfFileExists{microtype.sty}{% use microtype if available
  \usepackage[]{microtype}
  \UseMicrotypeSet[protrusion]{basicmath} % disable protrusion for tt fonts
}{}
\makeatletter
\@ifundefined{KOMAClassName}{% if non-KOMA class
  \IfFileExists{parskip.sty}{%
    \usepackage{parskip}
  }{% else
    \setlength{\parindent}{0pt}
    \setlength{\parskip}{6pt plus 2pt minus 1pt}}
}{% if KOMA class
  \KOMAoptions{parskip=half}}
\makeatother
\usepackage{xcolor}
\usepackage[margin=1in]{geometry}
\usepackage{color}
\usepackage{fancyvrb}
\newcommand{\VerbBar}{|}
\newcommand{\VERB}{\Verb[commandchars=\\\{\}]}
\DefineVerbatimEnvironment{Highlighting}{Verbatim}{commandchars=\\\{\}}
% Add ',fontsize=\small' for more characters per line
\usepackage{framed}
\definecolor{shadecolor}{RGB}{248,248,248}
\newenvironment{Shaded}{\begin{snugshade}}{\end{snugshade}}
\newcommand{\AlertTok}[1]{\textcolor[rgb]{0.94,0.16,0.16}{#1}}
\newcommand{\AnnotationTok}[1]{\textcolor[rgb]{0.56,0.35,0.01}{\textbf{\textit{#1}}}}
\newcommand{\AttributeTok}[1]{\textcolor[rgb]{0.13,0.29,0.53}{#1}}
\newcommand{\BaseNTok}[1]{\textcolor[rgb]{0.00,0.00,0.81}{#1}}
\newcommand{\BuiltInTok}[1]{#1}
\newcommand{\CharTok}[1]{\textcolor[rgb]{0.31,0.60,0.02}{#1}}
\newcommand{\CommentTok}[1]{\textcolor[rgb]{0.56,0.35,0.01}{\textit{#1}}}
\newcommand{\CommentVarTok}[1]{\textcolor[rgb]{0.56,0.35,0.01}{\textbf{\textit{#1}}}}
\newcommand{\ConstantTok}[1]{\textcolor[rgb]{0.56,0.35,0.01}{#1}}
\newcommand{\ControlFlowTok}[1]{\textcolor[rgb]{0.13,0.29,0.53}{\textbf{#1}}}
\newcommand{\DataTypeTok}[1]{\textcolor[rgb]{0.13,0.29,0.53}{#1}}
\newcommand{\DecValTok}[1]{\textcolor[rgb]{0.00,0.00,0.81}{#1}}
\newcommand{\DocumentationTok}[1]{\textcolor[rgb]{0.56,0.35,0.01}{\textbf{\textit{#1}}}}
\newcommand{\ErrorTok}[1]{\textcolor[rgb]{0.64,0.00,0.00}{\textbf{#1}}}
\newcommand{\ExtensionTok}[1]{#1}
\newcommand{\FloatTok}[1]{\textcolor[rgb]{0.00,0.00,0.81}{#1}}
\newcommand{\FunctionTok}[1]{\textcolor[rgb]{0.13,0.29,0.53}{\textbf{#1}}}
\newcommand{\ImportTok}[1]{#1}
\newcommand{\InformationTok}[1]{\textcolor[rgb]{0.56,0.35,0.01}{\textbf{\textit{#1}}}}
\newcommand{\KeywordTok}[1]{\textcolor[rgb]{0.13,0.29,0.53}{\textbf{#1}}}
\newcommand{\NormalTok}[1]{#1}
\newcommand{\OperatorTok}[1]{\textcolor[rgb]{0.81,0.36,0.00}{\textbf{#1}}}
\newcommand{\OtherTok}[1]{\textcolor[rgb]{0.56,0.35,0.01}{#1}}
\newcommand{\PreprocessorTok}[1]{\textcolor[rgb]{0.56,0.35,0.01}{\textit{#1}}}
\newcommand{\RegionMarkerTok}[1]{#1}
\newcommand{\SpecialCharTok}[1]{\textcolor[rgb]{0.81,0.36,0.00}{\textbf{#1}}}
\newcommand{\SpecialStringTok}[1]{\textcolor[rgb]{0.31,0.60,0.02}{#1}}
\newcommand{\StringTok}[1]{\textcolor[rgb]{0.31,0.60,0.02}{#1}}
\newcommand{\VariableTok}[1]{\textcolor[rgb]{0.00,0.00,0.00}{#1}}
\newcommand{\VerbatimStringTok}[1]{\textcolor[rgb]{0.31,0.60,0.02}{#1}}
\newcommand{\WarningTok}[1]{\textcolor[rgb]{0.56,0.35,0.01}{\textbf{\textit{#1}}}}
\usepackage{graphicx}
\makeatletter
\def\maxwidth{\ifdim\Gin@nat@width>\linewidth\linewidth\else\Gin@nat@width\fi}
\def\maxheight{\ifdim\Gin@nat@height>\textheight\textheight\else\Gin@nat@height\fi}
\makeatother
% Scale images if necessary, so that they will not overflow the page
% margins by default, and it is still possible to overwrite the defaults
% using explicit options in \includegraphics[width, height, ...]{}
\setkeys{Gin}{width=\maxwidth,height=\maxheight,keepaspectratio}
% Set default figure placement to htbp
\makeatletter
\def\fps@figure{htbp}
\makeatother
\setlength{\emergencystretch}{3em} % prevent overfull lines
\providecommand{\tightlist}{%
  \setlength{\itemsep}{0pt}\setlength{\parskip}{0pt}}
\setcounter{secnumdepth}{-\maxdimen} % remove section numbering
\ifLuaTeX
  \usepackage{selnolig}  % disable illegal ligatures
\fi
\IfFileExists{bookmark.sty}{\usepackage{bookmark}}{\usepackage{hyperref}}
\IfFileExists{xurl.sty}{\usepackage{xurl}}{} % add URL line breaks if available
\urlstyle{same}
\hypersetup{
  pdftitle={Taller 1},
  pdfauthor={Andrés Díaz, Yilmer Palacios},
  hidelinks,
  pdfcreator={LaTeX via pandoc}}

\title{Taller 1}
\author{Andrés Díaz, Yilmer Palacios}
\date{2024-02-01}

\begin{document}
\maketitle

\hypertarget{primer-punto}{%
\subsection{1 Primer Punto}\label{primer-punto}}

\hypertarget{definan-una-semilla-para-trabajar-durante-el-script.-respondan-por-quuxe9-es-importante-definir-una-semilla}{%
\subsubsection{1.1) Definan una semilla para trabajar durante el script.
Respondan: ¿Por qué es importante definir una
semilla?}\label{definan-una-semilla-para-trabajar-durante-el-script.-respondan-por-quuxe9-es-importante-definir-una-semilla}}

Respuesta: Según R-Coder.com, una semilla es el iniciador de un
generador de números pseudoaleatorios, que son utilizados para la
simulación de distribuciones de probabilidad que sean requeridas por el
algorítmo. Al ser pseudoaleatorios, los resultados obtenidos a partir de
estos números generados son replicables siempre y cuando se utilice el
mismo generador y la misma semilla.

Es importante definir una semilla en un algoritmo que vaya a utilizar un
generador de números con distribuciones de probabilidad ya que permite
que terceros puedan llegar al mismo resultado con el algorítmo y los
datos utilizados. Si no se definiera semilla, los resultados cambiarían
cada vez que se ejecutara el código, dada la naturaleza de los datos que
se está utilizando.

Para este ejercicio, se definirá la semilla usando la fecha en la que se
comenzó a trabajar en este taller.

\begin{Shaded}
\begin{Highlighting}[]
\FunctionTok{set.seed}\NormalTok{(}\DecValTok{31012024}\NormalTok{)}
\end{Highlighting}
\end{Shaded}

\hypertarget{primero-creen-una-lista-con-nuxfameros-secuenciales-de-1-en-1-desde-el-1-hasta-el-50.-luego-creen-tres-3-listas-diferentes-que-contengan-respectivamente-una-variable-numuxe9rica-de-clase-int-que-se-distribuya-de-forma-uniforme-entre-el-intervalo-5-a-50-una-lista-que-repita-el-caruxe1cter-auxf1os-y-una-lista-con-nombres-propios-aleatorios-de-personas-todas-las-cuatro-4-listas-deben-tener-el-mismo-tamauxf1o.}{%
\subsubsection{1.2) Primero, creen una lista con números secuenciales de
1 en 1 desde el 1 hasta el 50. Luego, creen tres (3) listas diferentes
que contengan respectivamente: una variable numérica de clase int que se
distribuya de forma uniforme entre el intervalo 5 a 50, una lista que
repita el carácter ``Años'' y una lista con nombres propios aleatorios
de personas, todas las cuatro (4) listas deben tener el mismo
tamaño.}\label{primero-creen-una-lista-con-nuxfameros-secuenciales-de-1-en-1-desde-el-1-hasta-el-50.-luego-creen-tres-3-listas-diferentes-que-contengan-respectivamente-una-variable-numuxe9rica-de-clase-int-que-se-distribuya-de-forma-uniforme-entre-el-intervalo-5-a-50-una-lista-que-repita-el-caruxe1cter-auxf1os-y-una-lista-con-nombres-propios-aleatorios-de-personas-todas-las-cuatro-4-listas-deben-tener-el-mismo-tamauxf1o.}}

\begin{Shaded}
\begin{Highlighting}[]
\FunctionTok{set.seed}\NormalTok{(}\DecValTok{31012024}\NormalTok{) }\CommentTok{\#La ejecutamos de nuevo para que tenga efecto sobre el runif}
\FunctionTok{rm}\NormalTok{(}\AttributeTok{list=}\FunctionTok{ls}\NormalTok{())   }\CommentTok{\# borramos environment}
\NormalTok{l\_num }\OtherTok{=} \FunctionTok{list}\NormalTok{(}\AttributeTok{numeros =} \FunctionTok{seq}\NormalTok{(}\DecValTok{1}\NormalTok{,}\DecValTok{50}\NormalTok{,}\DecValTok{1}\NormalTok{)) }\CommentTok{\# Creamos la lista de 1 en 1 hasta 50}
\NormalTok{l\_edad }\OtherTok{=} \FunctionTok{list}\NormalTok{(}\AttributeTok{edad=} \FunctionTok{round}\NormalTok{(}\FunctionTok{runif}\NormalTok{(}\DecValTok{50}\NormalTok{,}\DecValTok{5}\NormalTok{,}\DecValTok{50}\NormalTok{),}\DecValTok{0}\NormalTok{))}\CommentTok{\# Creamos la lista de edades con una distribución uniforme entre 5 y 50}
\NormalTok{l\_texto }\OtherTok{=} \FunctionTok{list}\NormalTok{(}\AttributeTok{variable=}\FunctionTok{rep}\NormalTok{(}\StringTok{"años"}\NormalTok{,}\DecValTok{50}\NormalTok{)) }\CommentTok{\# Creamos una lista con la palabra "años" repetida 50 veces}


\CommentTok{\# Para la lista de nombres vamos a crear dos vectores, uno con los nombres y otro con los apellidos de los integrantes del curso; luego usaremos la función sample para tomar de los vectores elementos aleatorios; finalmente, uniremos los nombres y apellidos aleatorios para formar la lista. Fuente: https://r{-}coder.com/funcion{-}sample{-}r/}

\NormalTok{rapellidos }\OtherTok{\textless{}{-}} \FunctionTok{c}\NormalTok{(}\StringTok{"Aguirre"}\NormalTok{, }\StringTok{"Arteaga"}\NormalTok{, }\StringTok{"Borda"}\NormalTok{, }\StringTok{"Caballero"}\NormalTok{, }\StringTok{"Carbonell"}\NormalTok{, }\StringTok{"Carvajal"}\NormalTok{, }\StringTok{"Criollo"}\NormalTok{, }\StringTok{"Diaz"}\NormalTok{, }\StringTok{"Duquino"}\NormalTok{, }\StringTok{"Escobar"}\NormalTok{, }\StringTok{"Fernandez"}\NormalTok{, }\StringTok{"Fonseca"}\NormalTok{, }\StringTok{"Galindo"}\NormalTok{, }\StringTok{"Gonzalez"}\NormalTok{, }\StringTok{"Hernandez"}\NormalTok{, }\StringTok{"Huertas"}\NormalTok{, }\StringTok{"Karaman"}\NormalTok{, }\StringTok{"Lasso"}\NormalTok{, }\StringTok{"Naranjo"}\NormalTok{, }\StringTok{"Navarrete"}\NormalTok{, }\StringTok{"Navarro"}\NormalTok{, }\StringTok{"Neusa"}\NormalTok{, }\StringTok{"Osorno"}\NormalTok{, }\StringTok{"Palacios"}\NormalTok{, }\StringTok{"Patiño"}\NormalTok{, }\StringTok{"Perdomo"}\NormalTok{, }\StringTok{"Prada"}\NormalTok{, }\StringTok{"Rueda"}\NormalTok{, }\StringTok{"Villamizar"}\NormalTok{, }\StringTok{"Viveros"}\NormalTok{, }\StringTok{"Zuluaga"}\NormalTok{, }\StringTok{"Valencia"}\NormalTok{, }\StringTok{"Gomez"}\NormalTok{, }\StringTok{"Bejarano"}\NormalTok{, }\StringTok{"Silva"}\NormalTok{, }\StringTok{"Ortega"}\NormalTok{, }\StringTok{"Huerfano"}\NormalTok{, }\StringTok{"Cardenas"}\NormalTok{, }\StringTok{"Barreto"}\NormalTok{, }\StringTok{"Carreño"}\NormalTok{, }\StringTok{"Rocha"}\NormalTok{, }\StringTok{"Correa"}\NormalTok{, }\StringTok{"Jimenez"}\NormalTok{, }\StringTok{"Barragan"}\NormalTok{, }\StringTok{"Rincon"}\NormalTok{, }\StringTok{"Ramirez"}\NormalTok{, }\StringTok{"Charry"}\NormalTok{, }\StringTok{"Munoz"}\NormalTok{, }\StringTok{"Pedreros"}\NormalTok{, }\StringTok{"Lian"}\NormalTok{, }\StringTok{"Jaramillo"}\NormalTok{, }\StringTok{"DeZulategi"}\NormalTok{, }\StringTok{"Amaya"}\NormalTok{, }\StringTok{"Corredor"}\NormalTok{, }\StringTok{"Ortiz"}\NormalTok{, }\StringTok{"Rodriguez"}\NormalTok{, }\StringTok{"Martinez"}\NormalTok{, }\StringTok{"Olmos"}\NormalTok{, }\StringTok{"Echeverri"}\NormalTok{, }\StringTok{"Arias"}\NormalTok{, }\StringTok{"Moreno"}\NormalTok{, }\StringTok{"Rojas"}\NormalTok{, }\StringTok{"Aguilar"}\NormalTok{, }\StringTok{"Cuellar"}\NormalTok{, }\StringTok{"Guacheta"}\NormalTok{, }\StringTok{"Vargas"}\NormalTok{, }\StringTok{"Muaoz"}\NormalTok{, }\StringTok{"DelCastillo"}\NormalTok{)}

\NormalTok{rnombres }\OtherTok{\textless{}{-}} \FunctionTok{c}\NormalTok{(}\StringTok{"Dennis"}\NormalTok{, }\StringTok{"Diana"}\NormalTok{, }\StringTok{"Luis"}\NormalTok{, }\StringTok{"Alejandra"}\NormalTok{, }\StringTok{"Henry"}\NormalTok{, }\StringTok{"Pharad"}\NormalTok{, }\StringTok{"Andres"}\NormalTok{, }\StringTok{"Edinson"}\NormalTok{, }\StringTok{"Yuri"}\NormalTok{, }\StringTok{"Jose"}\NormalTok{, }\StringTok{"Laura"}\NormalTok{, }\StringTok{"German"}\NormalTok{, }\StringTok{"Valentina"}\NormalTok{, }\StringTok{"Daniel"}\NormalTok{, }\StringTok{"Samuel"}\NormalTok{, }\StringTok{"Manuela"}\NormalTok{, }\StringTok{"Manuel"}\NormalTok{, }\StringTok{"Angella"}\NormalTok{, }\StringTok{"Yilmer"}\NormalTok{, }\StringTok{"Raul"}\NormalTok{, }\StringTok{"Gabriel"}\NormalTok{, }\StringTok{"Angela"}\NormalTok{, }\StringTok{"Camilo"}\NormalTok{, }\StringTok{"Santiago"}\NormalTok{, }\StringTok{"Julian"}\NormalTok{, }\StringTok{"Juanita"}\NormalTok{, }\StringTok{"Juan"}\NormalTok{, }\StringTok{"Silvia"}\NormalTok{, }\StringTok{"Francisco"}\NormalTok{, }\StringTok{"Estefania"}\NormalTok{, }\StringTok{"Johana"}\NormalTok{, }\StringTok{"Carolina"}\NormalTok{, }\StringTok{"Olegario"}\NormalTok{, }\StringTok{"Valentina"}\NormalTok{, }\StringTok{"Nicolas"}\NormalTok{, }\StringTok{"Fernando"}\NormalTok{, }\StringTok{"Alejandro"}\NormalTok{, }\StringTok{"Romina"}\NormalTok{, }\StringTok{"Camila"}\NormalTok{, }\StringTok{"Salomon"}\NormalTok{, }\StringTok{"Sofia"}\NormalTok{, }\StringTok{"Felipe"}\NormalTok{, }\StringTok{"Enrique"}\NormalTok{, }\StringTok{"Katherine"}\NormalTok{, }\StringTok{"Sebastian"}\NormalTok{, }\StringTok{"Alexis"}\NormalTok{, }\StringTok{"Maria"}\NormalTok{, }\StringTok{"Alexander"}\NormalTok{, }\StringTok{"David"}\NormalTok{, }\StringTok{"Paula"}\NormalTok{, }\StringTok{"Felipe"}\NormalTok{, }\StringTok{"Juliana"}\NormalTok{)}

\NormalTok{nom\_apellido }\OtherTok{\textless{}{-}} \FunctionTok{c}\NormalTok{(}\FunctionTok{paste}\NormalTok{(}\FunctionTok{sample}\NormalTok{(rnombres,}\DecValTok{50}\NormalTok{, }\AttributeTok{replace=}\ConstantTok{TRUE}\NormalTok{), }\FunctionTok{sample}\NormalTok{(rapellidos,}\DecValTok{50}\NormalTok{, }\AttributeTok{replace=}\ConstantTok{TRUE}\NormalTok{)))}

\CommentTok{\#Ahora si creamos la lista}

\NormalTok{l\_nombres }\OtherTok{=} \FunctionTok{list}\NormalTok{(nom\_apellido)}
\end{Highlighting}
\end{Shaded}

\hypertarget{creen-una-lista-en-la-que-cada-elemento-j-sea-la-concatenaciuxf3n-de-los-elementos-j-de-las-tres-listas-creadas-en-el-punto-anterior.-ordenen-yo-agreguen-caracteres-a-cada-elemento-de-la-lista-para-que-se-consolide-una-oraciuxf3n-con-orden-semuxe1ntico-que-refleje-la-edad-del-individuo.}{%
\subsubsection{1.3) Creen una lista en la que cada elemento j sea la
concatenación de los elementos j de las tres listas creadas en el punto
anterior. Ordenen y/o agreguen caracteres a cada elemento de la lista
para que se consolide una oración con orden semántico que refleje la
edad del
individuo.}\label{creen-una-lista-en-la-que-cada-elemento-j-sea-la-concatenaciuxf3n-de-los-elementos-j-de-las-tres-listas-creadas-en-el-punto-anterior.-ordenen-yo-agreguen-caracteres-a-cada-elemento-de-la-lista-para-que-se-consolide-una-oraciuxf3n-con-orden-semuxe1ntico-que-refleje-la-edad-del-individuo.}}

\begin{Shaded}
\begin{Highlighting}[]
\NormalTok{l\_oracion }\OtherTok{\textless{}{-}} \FunctionTok{c}\NormalTok{()}

\CommentTok{\#creamos un bucle que va desde 1 hasta el número de elementos (50), es cada paso del bucle se concatenarán con "paste" las listas y se guaradán en una posición de la lista "l\_oracion"}

\ControlFlowTok{for}\NormalTok{ (i }\ControlFlowTok{in} \DecValTok{1}\SpecialCharTok{:}\FunctionTok{length}\NormalTok{(l\_nombres[[}\DecValTok{1}\NormalTok{]]))\{}
  
\NormalTok{  l\_oracion[[}\DecValTok{1}\NormalTok{]][i] }\OtherTok{\textless{}{-}} \FunctionTok{paste}\NormalTok{(l\_nombres[[}\DecValTok{1}\NormalTok{]][i],}\StringTok{"tiene"}\NormalTok{,l\_edad[[}\DecValTok{1}\NormalTok{]][i],}
\NormalTok{                          l\_texto[[}\DecValTok{1}\NormalTok{]][i])    }
  
\NormalTok{\}}

\NormalTok{l\_oracion}
\end{Highlighting}
\end{Shaded}

\begin{verbatim}
## [[1]]
##  [1] "Camila Rincon tiene 6 años"        "Laura Aguilar tiene 49 años"      
##  [3] "Alejandro Viveros tiene 29 años"   "Daniel Muaoz tiene 40 años"       
##  [5] "Dennis Patiño tiene 38 años"       "Valentina Patiño tiene 11 años"   
##  [7] "Andres Guacheta tiene 5 años"      "Maria Arias tiene 22 años"        
##  [9] "Dennis Barreto tiene 29 años"      "Romina Navarrete tiene 16 años"   
## [11] "Diana Silva tiene 31 años"         "Valentina Rodriguez tiene 36 años"
## [13] "Andres Palacios tiene 35 años"     "Camilo Martinez tiene 24 años"    
## [15] "Juliana Arteaga tiene 38 años"     "Olegario Cardenas tiene 32 años"  
## [17] "Manuela Charry tiene 49 años"      "Juliana DeZulategi tiene 36 años" 
## [19] "Olegario Cuellar tiene 9 años"     "Juanita Arteaga tiene 38 años"    
## [21] "Alejandro Neusa tiene 30 años"     "David Diaz tiene 27 años"         
## [23] "Camila Jimenez tiene 10 años"      "Juanita Huertas tiene 35 años"    
## [25] "Juanita Diaz tiene 21 años"        "Diana Arias tiene 24 años"        
## [27] "Angela Galindo tiene 15 años"      "Raul Valencia tiene 31 años"      
## [29] "Carolina Valencia tiene 46 años"   "Maria Villamizar tiene 46 años"   
## [31] "Alejandro Charry tiene 47 años"    "Camila Guacheta tiene 21 años"    
## [33] "Pharad Echeverri tiene 40 años"    "Nicolas Echeverri tiene 27 años"  
## [35] "Yilmer Diaz tiene 26 años"         "Edinson Munoz tiene 38 años"      
## [37] "Jose Rojas tiene 8 años"           "Salomon Jaramillo tiene 14 años"  
## [39] "Paula Echeverri tiene 21 años"     "Alexander Rojas tiene 19 años"    
## [41] "Laura Neusa tiene 22 años"         "Juliana Galindo tiene 24 años"    
## [43] "Manuela Corredor tiene 21 años"    "Felipe Aguilar tiene 33 años"     
## [45] "Paula Martinez tiene 10 años"      "Raul Caballero tiene 30 años"     
## [47] "German Munoz tiene 39 años"        "Valentina Carbonell tiene 48 años"
## [49] "Silvia Galindo tiene 8 años"       "David Martinez tiene 29 años"
\end{verbatim}

\hypertarget{usando-un-loop-realicen-un-cuxf3digo-que-presente-print-la-edad-de-cada-uno-de-los-individuos-dentro-de-las-listas-pero-uxfanicamente-si-el-nombre-del-individuo-empieza-por-una-letra-distinta-de-j-y-la-edad-sea-distinta-de-un-nuxfamero-par.}{%
\subsubsection{1.4) Usando un loop realicen un código que presente
(print) la edad de cada uno de los individuos dentro de las listas, pero
únicamente si el nombre del individuo empieza por una letra distinta de
J y la edad sea distinta de un número
par.}\label{usando-un-loop-realicen-un-cuxf3digo-que-presente-print-la-edad-de-cada-uno-de-los-individuos-dentro-de-las-listas-pero-uxfanicamente-si-el-nombre-del-individuo-empieza-por-una-letra-distinta-de-j-y-la-edad-sea-distinta-de-un-nuxfamero-par.}}

\begin{Shaded}
\begin{Highlighting}[]
\DocumentationTok{\#\# Creamos un ciclo for que va desde 1 hasta el número de elementos de la lista oración (50), luego pasamos a dos codicionales, el primero evalúa si el nombre inicia por J y el segundo si el residuo de dividir la edad en 2 es mayor que cero, si lo es, el número es impar, a los elementos que pasen ambas condiciones se les hará "print".}

\ControlFlowTok{for}\NormalTok{ (j }\ControlFlowTok{in} \DecValTok{1}\SpecialCharTok{:}\FunctionTok{length}\NormalTok{(l\_oracion[[}\DecValTok{1}\NormalTok{]]))\{}
  
  
  \ControlFlowTok{if}\NormalTok{(}\FunctionTok{substring}\NormalTok{(l\_oracion[[}\DecValTok{1}\NormalTok{]][j],}\DecValTok{1}\NormalTok{,}\DecValTok{1}\NormalTok{)}\SpecialCharTok{!=}\StringTok{"J"}\NormalTok{)\{}
    
    \ControlFlowTok{if}\NormalTok{(l\_edad[[}\DecValTok{1}\NormalTok{]][j]}\SpecialCharTok{\%\%}\DecValTok{2}\SpecialCharTok{\textgreater{}}\DecValTok{0}\NormalTok{)\{}
      
      \FunctionTok{print}\NormalTok{(l\_oracion[[}\DecValTok{1}\NormalTok{]][j])}
      
\NormalTok{    \}}
    
\NormalTok{  \}}

\NormalTok{\}}
\end{Highlighting}
\end{Shaded}

\begin{verbatim}
## [1] "Laura Aguilar tiene 49 años"
## [1] "Alejandro Viveros tiene 29 años"
## [1] "Valentina Patiño tiene 11 años"
## [1] "Andres Guacheta tiene 5 años"
## [1] "Dennis Barreto tiene 29 años"
## [1] "Diana Silva tiene 31 años"
## [1] "Andres Palacios tiene 35 años"
## [1] "Manuela Charry tiene 49 años"
## [1] "Olegario Cuellar tiene 9 años"
## [1] "David Diaz tiene 27 años"
## [1] "Angela Galindo tiene 15 años"
## [1] "Raul Valencia tiene 31 años"
## [1] "Alejandro Charry tiene 47 años"
## [1] "Camila Guacheta tiene 21 años"
## [1] "Nicolas Echeverri tiene 27 años"
## [1] "Paula Echeverri tiene 21 años"
## [1] "Alexander Rojas tiene 19 años"
## [1] "Manuela Corredor tiene 21 años"
## [1] "Felipe Aguilar tiene 33 años"
## [1] "German Munoz tiene 39 años"
## [1] "David Martinez tiene 29 años"
\end{verbatim}

\hypertarget{programen-una-funciuxf3n-que-tome-como-entrada-una-lista-con-valores-numuxe9ricos-y-que-su-output-sea-el-promedio-de-los-valores-de-la-lista-y-la-desviaciuxf3n-estuxe1ndar-asociada-a-la-misma-muestra.-usando-esta-funciuxf3n-respondan-cuuxe1l-es-la-edad-promedio-de-su-lista-cuuxe1l-es-la-desviaciuxf3n-estuxe1ndar}{%
\subsubsection{1.5) Programen una función que tome como entrada una
lista con valores numéricos y que su output sea el promedio de los
valores de la lista y la desviación estándar asociada a la misma
muestra. Usando esta función respondan: ¿Cuál es la edad promedio de su
lista? ¿Cuál es la desviación
estándar?}\label{programen-una-funciuxf3n-que-tome-como-entrada-una-lista-con-valores-numuxe9ricos-y-que-su-output-sea-el-promedio-de-los-valores-de-la-lista-y-la-desviaciuxf3n-estuxe1ndar-asociada-a-la-misma-muestra.-usando-esta-funciuxf3n-respondan-cuuxe1l-es-la-edad-promedio-de-su-lista-cuuxe1l-es-la-desviaciuxf3n-estuxe1ndar}}

\begin{Shaded}
\begin{Highlighting}[]
\NormalTok{  funcion\_param\_stat }\OtherTok{\textless{}{-}} \ControlFlowTok{function}\NormalTok{(lista\_edad)\{}
    
\NormalTok{    media }\OtherTok{\textless{}{-}} \FunctionTok{mean}\NormalTok{(lista\_edad)}
\NormalTok{    des\_est }\OtherTok{\textless{}{-}} \FunctionTok{sd}\NormalTok{(lista\_edad)}
    \FunctionTok{print}\NormalTok{(}\FunctionTok{paste}\NormalTok{(}\StringTok{"La edad promedio es"}\NormalTok{,media))}
    \FunctionTok{print}\NormalTok{(}\FunctionTok{paste}\NormalTok{(}\StringTok{"La desviación estándar es"}\NormalTok{,}\FunctionTok{signif}\NormalTok{(des\_est,}\DecValTok{4}\NormalTok{)))}
    
\NormalTok{  \}}

\FunctionTok{funcion\_param\_stat}\NormalTok{(l\_edad[[}\DecValTok{1}\NormalTok{]])}
\end{Highlighting}
\end{Shaded}

\begin{verbatim}
## [1] "La edad promedio es 27.66"
## [1] "La desviación estándar es 12.3"
\end{verbatim}

\hypertarget{programen-una-funciuxf3n-que-tome-como-entrada-una-lista-con-valores-numuxe9ricos-y-estandarice-los-valores.-es-decir-que-los-transforme-a-una-normal-estuxe1ndar-los-datos.-apliquen-las-funciones-que-desarrollaron-en-el-literal-1.5-dentro-de-la-funciuxf3n-que-propongan-en-este-literal}{%
\subsubsection{1.6) Programen una función que tome como entrada una
lista con valores numéricos y estandarice los valores. Es decir, que los
transforme a una normal estándar los datos. Apliquen las funciones que
desarrollaron en el literal 1.5) dentro de la función que propongan en
este
literal}\label{programen-una-funciuxf3n-que-tome-como-entrada-una-lista-con-valores-numuxe9ricos-y-estandarice-los-valores.-es-decir-que-los-transforme-a-una-normal-estuxe1ndar-los-datos.-apliquen-las-funciones-que-desarrollaron-en-el-literal-1.5-dentro-de-la-funciuxf3n-que-propongan-en-este-literal}}

\begin{Shaded}
\begin{Highlighting}[]
\NormalTok{  funcion\_norm\_estan }\OtherTok{\textless{}{-}} \ControlFlowTok{function}\NormalTok{(listaval)\{}
    
\NormalTok{    media }\OtherTok{\textless{}{-}} \FunctionTok{mean}\NormalTok{(listaval)}
\NormalTok{    des\_est }\OtherTok{\textless{}{-}} \FunctionTok{sd}\NormalTok{(listaval)}

\NormalTok{    dist\_estandar }\OtherTok{\textless{}{-}} \FunctionTok{c}\NormalTok{()}
    
    \ControlFlowTok{for}\NormalTok{ (i }\ControlFlowTok{in} \DecValTok{1}\SpecialCharTok{:}\FunctionTok{length}\NormalTok{(listaval))\{}
    
\NormalTok{    dist\_estandar[i] }\OtherTok{\textless{}{-}}\NormalTok{ (listaval[i]}\SpecialCharTok{{-}}\NormalTok{media)}\SpecialCharTok{/}\NormalTok{des\_est}
    
\NormalTok{    \}}
    
    \FunctionTok{print}\NormalTok{(dist\_estandar)}
   
\NormalTok{  \}}

\NormalTok{dist\_estandar }\OtherTok{\textless{}{-}} \FunctionTok{funcion\_norm\_estan}\NormalTok{(l\_edad[[}\DecValTok{1}\NormalTok{]])}
\end{Highlighting}
\end{Shaded}

\begin{verbatim}
##  [1] -1.7609744  1.7349582  0.1089430  1.0032514  0.8406498 -1.3544706
##  [7] -1.8422752 -0.4601623  0.1089430 -0.9479668  0.2715445  0.6780483
## [13]  0.5967476 -0.2975608  0.8406498  0.3528453  1.7349582  0.6780483
## [19] -1.5170721  0.8406498  0.1902438 -0.0536585 -1.4357714  0.5967476
## [25] -0.5414630 -0.2975608 -1.0292676  0.2715445  1.4910559  1.4910559
## [31]  1.5723567 -0.5414630  1.0032514 -0.0536585 -0.1349593  0.8406498
## [37] -1.5983729 -1.1105684 -0.5414630 -0.7040646 -0.4601623 -0.2975608
## [43] -0.5414630  0.4341460 -1.4357714  0.1902438  0.9219506  1.6536574
## [49] -1.5983729  0.1089430
\end{verbatim}

\hypertarget{por-otra-parte-generen-una-lista-de-listas-llamada-outcomes_nominales.-esta-lista-contendruxe1-3-vectores-de-50-observaciones-cada-uno-con-los-outcomes-de-interuxe9s-salario-uxedndice-de-salud-experiencia-laboral.-para-esto-generen-para-cada-vector-valores-numuxe9ricos-de-clase-float-basado-en-una-distribuciuxf3n-normal-estuxe1ndar-con-media-0-y-varianza-1.}{%
\subsubsection{1.7) Por otra parte, generen una lista de listas llamada
outcomes\_nominales. Esta lista contendrá 3 vectores de 50 observaciones
cada uno con los outcomes de interés: salario, índice de salud,
experiencia laboral. Para esto, generen para cada vector valores
numéricos de clase float basado en una distribución normal estándar con
media 0 y varianza
1.}\label{por-otra-parte-generen-una-lista-de-listas-llamada-outcomes_nominales.-esta-lista-contendruxe1-3-vectores-de-50-observaciones-cada-uno-con-los-outcomes-de-interuxe9s-salario-uxedndice-de-salud-experiencia-laboral.-para-esto-generen-para-cada-vector-valores-numuxe9ricos-de-clase-float-basado-en-una-distribuciuxf3n-normal-estuxe1ndar-con-media-0-y-varianza-1.}}

\begin{Shaded}
\begin{Highlighting}[]
\FunctionTok{set.seed}\NormalTok{(}\DecValTok{31012024}\NormalTok{)}
\NormalTok{salario }\OtherTok{\textless{}{-}} \FunctionTok{rnorm}\NormalTok{(}\DecValTok{50}\NormalTok{,}\DecValTok{0}\NormalTok{,}\DecValTok{1}\NormalTok{)}
\NormalTok{ind\_salud }\OtherTok{\textless{}{-}} \FunctionTok{rnorm}\NormalTok{(}\DecValTok{50}\NormalTok{,}\DecValTok{0}\NormalTok{,}\DecValTok{1}\NormalTok{)}
\NormalTok{exp\_laboral }\OtherTok{\textless{}{-}} \FunctionTok{rnorm}\NormalTok{(}\DecValTok{50}\NormalTok{,}\DecValTok{0}\NormalTok{,}\DecValTok{1}\NormalTok{)}

\NormalTok{outcomes\_nominales }\OtherTok{\textless{}{-}} \FunctionTok{list}\NormalTok{(}\AttributeTok{Salario =}\NormalTok{ salario,}\AttributeTok{IndiceSalud =}\NormalTok{ ind\_salud, }\AttributeTok{ExpLaboral =}\NormalTok{ exp\_laboral)}
\FunctionTok{print}\NormalTok{(outcomes\_nominales)}
\end{Highlighting}
\end{Shaded}

\begin{verbatim}
## $Salario
##  [1] -1.91416079  0.08283508  0.60387299 -2.41213551  0.06383401  0.19480237
##  [7]  0.46030633  0.59475467  2.19532204 -1.40370623  0.14190947 -1.20862880
## [13] -0.39271258 -0.76049622  1.34540467  1.46010798  0.78195863 -0.05794777
## [19] -1.55772948 -0.38427497 -0.30211594 -0.38658020 -1.23173010  0.66510870
## [25] -1.46091704 -0.59775726  1.79699371  1.80880576 -0.85514235  1.23510008
## [31]  0.13529639  1.63720832 -0.18216793  0.11242070 -0.09909349 -0.30273958
## [37] -0.34165724  3.54663044  0.34843065  0.66299703  0.32769197  0.53503017
## [43]  0.82915328  0.47189576  1.10595962 -0.51280806  0.49014971 -1.86663771
## [49]  0.60413448 -0.39408111
## 
## $IndiceSalud
##  [1] -1.388088237  0.113564205 -0.391351305  0.168781987  0.004735466
##  [6]  0.489238913  0.341522750  1.468029567 -0.368587272  0.129880746
## [11]  2.333766030 -0.880164560  0.523374668  1.818628231 -0.566250997
## [16]  1.959629655  1.187044268 -1.452526484 -0.642244988 -0.322394590
## [21] -0.165078338 -0.565036365  0.740957706 -0.195629731  0.805880574
## [26] -0.127574338  0.509536329  0.621140869  0.916252096 -0.762888720
## [31] -0.532742753  0.400304175  1.132465071  0.529173913 -0.087158133
## [36]  0.883525585 -1.492477526  0.296693762 -0.478331572 -0.695289255
## [41] -0.311569947 -1.594083429  0.166684231 -0.857302800 -0.245187888
## [46] -1.903302208  0.485943384 -0.519674229  0.854043886 -1.457759585
## 
## $ExpLaboral
##  [1]  0.297300659  0.372693193  0.103886105 -0.293698669 -1.468215527
##  [6] -0.570178265  1.096950099 -1.144961795 -0.746812669  1.327276595
## [11] -0.919121509 -0.964707460 -0.384276547  0.125925023 -0.950705622
## [16]  1.659851435  0.005855219  0.046903205 -0.291512620 -0.036997326
## [21]  0.990550383  0.822904884 -2.711220988 -1.237013300  0.902699590
## [26]  1.620900029  0.831470677  0.469134755  1.602089729 -0.532298390
## [31] -1.072937527 -0.272019289 -0.477099217  0.781683806  0.113061748
## [36] -0.800484756  0.074146031 -0.020136954  0.612494007  0.510199231
## [41] -0.559683996 -2.250110313  1.005131196 -0.891300623  0.605136784
## [46] -0.999962139 -0.280055914  0.880141053 -1.507191621 -0.255060619
\end{verbatim}

\hypertarget{creen-una-funciuxf3n-que-transforme-una-lista-en-una-matriz.-para-esto-la-funciuxf3n-debe-tomar-como-input-una-lista-y-debe-tener-como-output-una-matriz-x-que-concatene-los-datos-de-esta-lista-y-un-vector-de-1s.}{%
\subsubsection{1.8) Creen una función que transforme una lista en una
matriz. Para esto, la función debe tomar como input una lista y debe
tener como output una matriz X que concatene los datos de esta lista y
un vector de
1´s.}\label{creen-una-funciuxf3n-que-transforme-una-lista-en-una-matriz.-para-esto-la-funciuxf3n-debe-tomar-como-input-una-lista-y-debe-tener-como-output-una-matriz-x-que-concatene-los-datos-de-esta-lista-y-un-vector-de-1s.}}

\begin{Shaded}
\begin{Highlighting}[]
\NormalTok{  funcion\_lista\_matriz }\OtherTok{\textless{}{-}} \ControlFlowTok{function}\NormalTok{(lista)\{}
    
    \CommentTok{\#Input}
      \CommentTok{\#Vector de unos con la longitud de la edad}
      \CommentTok{\#Vector de la edad con la edad    }
\NormalTok{    vector\_unos }\OtherTok{\textless{}{-}} \FunctionTok{matrix}\NormalTok{(}\FunctionTok{rep}\NormalTok{(}\DecValTok{1}\NormalTok{,}\FunctionTok{length}\NormalTok{(lista[[}\DecValTok{1}\NormalTok{]])),}\AttributeTok{nrow=}\FunctionTok{length}\NormalTok{(l\_edad[[}\DecValTok{1}\NormalTok{]]))}
\NormalTok{    matriz }\OtherTok{\textless{}{-}} \FunctionTok{cbind}\NormalTok{(lista,vector\_unos)}
    
    \FunctionTok{colnames}\NormalTok{(matriz)[}\DecValTok{1}\NormalTok{] }\OtherTok{\textless{}{-}} \StringTok{"Edad"}
    \FunctionTok{colnames}\NormalTok{(matriz)[}\DecValTok{2}\NormalTok{] }\OtherTok{\textless{}{-}} \StringTok{"Constante"}
\NormalTok{    matriz}
    
\NormalTok{  \}}

\NormalTok{matriz }\OtherTok{=} \FunctionTok{funcion\_lista\_matriz}\NormalTok{(l\_edad[[}\DecValTok{1}\NormalTok{]])}
\FunctionTok{print}\NormalTok{(matriz)}
\end{Highlighting}
\end{Shaded}

\begin{verbatim}
##       Edad Constante
##  [1,]    6         1
##  [2,]   49         1
##  [3,]   29         1
##  [4,]   40         1
##  [5,]   38         1
##  [6,]   11         1
##  [7,]    5         1
##  [8,]   22         1
##  [9,]   29         1
## [10,]   16         1
## [11,]   31         1
## [12,]   36         1
## [13,]   35         1
## [14,]   24         1
## [15,]   38         1
## [16,]   32         1
## [17,]   49         1
## [18,]   36         1
## [19,]    9         1
## [20,]   38         1
## [21,]   30         1
## [22,]   27         1
## [23,]   10         1
## [24,]   35         1
## [25,]   21         1
## [26,]   24         1
## [27,]   15         1
## [28,]   31         1
## [29,]   46         1
## [30,]   46         1
## [31,]   47         1
## [32,]   21         1
## [33,]   40         1
## [34,]   27         1
## [35,]   26         1
## [36,]   38         1
## [37,]    8         1
## [38,]   14         1
## [39,]   21         1
## [40,]   19         1
## [41,]   22         1
## [42,]   24         1
## [43,]   21         1
## [44,]   33         1
## [45,]   10         1
## [46,]   30         1
## [47,]   39         1
## [48,]   48         1
## [49,]    8         1
## [50,]   29         1
\end{verbatim}

\begin{Shaded}
\begin{Highlighting}[]
\FunctionTok{class}\NormalTok{(matriz)}
\end{Highlighting}
\end{Shaded}

\begin{verbatim}
## [1] "matrix" "array"
\end{verbatim}

\hypertarget{a-partir-de-la-funciuxf3n-anterior-consoliden-una-matriz-x-con-la-edad-de-los-individuos-estandarizada-y-un-vector-de-1s-asociado-a-una-constante.}{%
\subsubsection{1.9) A partir de la función anterior consoliden una
matriz X con la edad de los individuos estandarizada y un vector de 1´s
asociado a una
constante.}\label{a-partir-de-la-funciuxf3n-anterior-consoliden-una-matriz-x-con-la-edad-de-los-individuos-estandarizada-y-un-vector-de-1s-asociado-a-una-constante.}}

\begin{Shaded}
\begin{Highlighting}[]
\NormalTok{vect\_edad\_estandar }\OtherTok{\textless{}{-}} \FunctionTok{funcion\_norm\_estan}\NormalTok{(l\_edad[[}\DecValTok{1}\NormalTok{]])}
\end{Highlighting}
\end{Shaded}

\begin{verbatim}
##  [1] -1.7609744  1.7349582  0.1089430  1.0032514  0.8406498 -1.3544706
##  [7] -1.8422752 -0.4601623  0.1089430 -0.9479668  0.2715445  0.6780483
## [13]  0.5967476 -0.2975608  0.8406498  0.3528453  1.7349582  0.6780483
## [19] -1.5170721  0.8406498  0.1902438 -0.0536585 -1.4357714  0.5967476
## [25] -0.5414630 -0.2975608 -1.0292676  0.2715445  1.4910559  1.4910559
## [31]  1.5723567 -0.5414630  1.0032514 -0.0536585 -0.1349593  0.8406498
## [37] -1.5983729 -1.1105684 -0.5414630 -0.7040646 -0.4601623 -0.2975608
## [43] -0.5414630  0.4341460 -1.4357714  0.1902438  0.9219506  1.6536574
## [49] -1.5983729  0.1089430
\end{verbatim}

\begin{Shaded}
\begin{Highlighting}[]
\NormalTok{matriz\_edad\_estandar }\OtherTok{\textless{}{-}} \FunctionTok{funcion\_lista\_matriz}\NormalTok{(vect\_edad\_estandar)}
\FunctionTok{colnames}\NormalTok{(matriz\_edad\_estandar)[}\DecValTok{1}\NormalTok{] }\OtherTok{\textless{}{-}} \StringTok{"Edad estándar"}
\FunctionTok{print}\NormalTok{(matriz\_edad\_estandar)}
\end{Highlighting}
\end{Shaded}

\begin{verbatim}
##       Edad estándar Constante
##  [1,]    -1.7609744         1
##  [2,]     1.7349582         1
##  [3,]     0.1089430         1
##  [4,]     1.0032514         1
##  [5,]     0.8406498         1
##  [6,]    -1.3544706         1
##  [7,]    -1.8422752         1
##  [8,]    -0.4601623         1
##  [9,]     0.1089430         1
## [10,]    -0.9479668         1
## [11,]     0.2715445         1
## [12,]     0.6780483         1
## [13,]     0.5967476         1
## [14,]    -0.2975608         1
## [15,]     0.8406498         1
## [16,]     0.3528453         1
## [17,]     1.7349582         1
## [18,]     0.6780483         1
## [19,]    -1.5170721         1
## [20,]     0.8406498         1
## [21,]     0.1902438         1
## [22,]    -0.0536585         1
## [23,]    -1.4357714         1
## [24,]     0.5967476         1
## [25,]    -0.5414630         1
## [26,]    -0.2975608         1
## [27,]    -1.0292676         1
## [28,]     0.2715445         1
## [29,]     1.4910559         1
## [30,]     1.4910559         1
## [31,]     1.5723567         1
## [32,]    -0.5414630         1
## [33,]     1.0032514         1
## [34,]    -0.0536585         1
## [35,]    -0.1349593         1
## [36,]     0.8406498         1
## [37,]    -1.5983729         1
## [38,]    -1.1105684         1
## [39,]    -0.5414630         1
## [40,]    -0.7040646         1
## [41,]    -0.4601623         1
## [42,]    -0.2975608         1
## [43,]    -0.5414630         1
## [44,]     0.4341460         1
## [45,]    -1.4357714         1
## [46,]     0.1902438         1
## [47,]     0.9219506         1
## [48,]     1.6536574         1
## [49,]    -1.5983729         1
## [50,]     0.1089430         1
\end{verbatim}

\hypertarget{segundo-punto}{%
\subsection{Segundo punto}\label{segundo-punto}}

\hypertarget{programen-una-funciuxf3n-que-tome-como-input-una-matriz-x-estocuxe1stica-de-rango-completo-y-un-vector-yi-posteriormente-el-output-debe-corresponder-a-una-estimaciuxf3n-puntual-del-estimador-ux3b21-de-muxednimos-cuadrados-ordinarios-mco-para-la-muestra-y-a-su-error-estuxe1ndar-asociado-ux3c3ux3b2.}{%
\subsubsection{2.1) Programen una función que tome como input una matriz
X estocástica de rango completo y un vector yi, posteriormente, el
output debe corresponder a una estimación puntual del estimador (β1) de
Mínimos Cuadrados Ordinarios (MCO) para la muestra y a su error estándar
asociado
(σβ).}\label{programen-una-funciuxf3n-que-tome-como-input-una-matriz-x-estocuxe1stica-de-rango-completo-y-un-vector-yi-posteriormente-el-output-debe-corresponder-a-una-estimaciuxf3n-puntual-del-estimador-ux3b21-de-muxednimos-cuadrados-ordinarios-mco-para-la-muestra-y-a-su-error-estuxe1ndar-asociado-ux3c3ux3b2.}}

\begin{Shaded}
\begin{Highlighting}[]
\CommentTok{\#Creamos dos funciones exactas salvo que la primera devuelve el estimado beta B y la segunda su error estándar.}

\NormalTok{funcion\_MCO\_beta1 }\OtherTok{\textless{}{-}} \ControlFlowTok{function}\NormalTok{(matriz\_in,y\_in)\{}
  
  \CommentTok{\#Input}
    \CommentTok{\# Matriz\_in: Matriz estocástica de rango completo}
    \CommentTok{\# y\_in: Vector y\_i}
  
  \CommentTok{\#Fórmula 1}
    \CommentTok{\# Beta\_1 = (t(X)*X)\^{}{-}1*(t(X)*y\_i)}
    
\NormalTok{  beta\_1 }\OtherTok{\textless{}{-}} \FunctionTok{solve}\NormalTok{(}\FunctionTok{t}\NormalTok{(matriz\_in) }\SpecialCharTok{\%*\%}\NormalTok{ matriz\_in) }\SpecialCharTok{\%*\%} 
\NormalTok{              (}\FunctionTok{t}\NormalTok{(matriz\_in) }\SpecialCharTok{\%*\%}\NormalTok{   y\_in)}
  
  \CommentTok{\#Fórmula 2}
    \CommentTok{\# ee(Beta\_1) = (Var(error)/STC)\^{}0.5}
      
      \CommentTok{\# Var(error) = SRC/(n{-}k{-}1) \#Varianza del error}
        \CommentTok{\#SRC = suma((y\_pred {-} y\_in)\^{}2) \#Sumatoria de los residuales al cuadrado}
        \CommentTok{\# k = 1 porque se van a hacer regresiones lineales simples}
        \CommentTok{\#y\_pred = X*beta\_1 \#Resultado predicho}
      
      \CommentTok{\# STC = suma((y\_in {-} mean(y\_in))\^{}2) \#Suma Total de Cuadrados}
    
\NormalTok{    y\_pred }\OtherTok{\textless{}{-}}\NormalTok{ matriz\_in  }\SpecialCharTok{\%*\%}\NormalTok{ beta\_1}
\NormalTok{    SRC }\OtherTok{\textless{}{-}} \FunctionTok{sum}\NormalTok{((y\_pred }\SpecialCharTok{{-}}\NormalTok{ y\_in)}\SpecialCharTok{\^{}}\DecValTok{2}\NormalTok{)}
\NormalTok{    Var\_error }\OtherTok{\textless{}{-}}\NormalTok{ SRC}\SpecialCharTok{/}\NormalTok{(}\FunctionTok{length}\NormalTok{(y\_in)}\SpecialCharTok{{-}}\DecValTok{1{-}1}\NormalTok{)}
\NormalTok{    STC }\OtherTok{\textless{}{-}} \FunctionTok{sum}\NormalTok{((y\_in }\SpecialCharTok{{-}} \FunctionTok{mean}\NormalTok{(y\_in))}\SpecialCharTok{\^{}}\DecValTok{2}\NormalTok{)}
    
\NormalTok{    eeBeta\_1 }\OtherTok{\textless{}{-}}\NormalTok{ (Var\_error}\SpecialCharTok{/}\NormalTok{STC)}\SpecialCharTok{\^{}}\FloatTok{0.5}

  \CommentTok{\#Output}
    \CommentTok{\# Beta\_1}
\NormalTok{    beta\_1[}\DecValTok{1}\NormalTok{]}
    
\NormalTok{\}}

\NormalTok{funcion\_MCO\_ee\_beta1 }\OtherTok{\textless{}{-}} \ControlFlowTok{function}\NormalTok{(matriz\_in,y\_in)\{}
  
  \CommentTok{\#Input}
    \CommentTok{\# Matriz\_in: Matriz estocástica de rango completo}
    \CommentTok{\# y\_in: Vector y\_i}
  
  \CommentTok{\#Fórmula 1}
  \CommentTok{\# Beta\_1 = (t(X)*X)\^{}{-}1*(t(X)*y\_i)}
    
\NormalTok{  beta\_1 }\OtherTok{\textless{}{-}} \FunctionTok{solve}\NormalTok{(}\FunctionTok{t}\NormalTok{(matriz\_in) }\SpecialCharTok{\%*\%}\NormalTok{ matriz\_in) }\SpecialCharTok{\%*\%} 
\NormalTok{              (}\FunctionTok{t}\NormalTok{(matriz\_in) }\SpecialCharTok{\%*\%}\NormalTok{   y\_in)}
  \CommentTok{\#Fórmula 2}
    \CommentTok{\# ee(Beta\_1) = (Var(error)/STC)\^{}0.5}
      
      \CommentTok{\# Var(error) = SRC/(n{-}k{-}1) \#Varianza del error}
        \CommentTok{\#SRC = suma((y\_pred {-} y\_in)\^{}2) \#Sumatoria de los residuales al           cuadrado}
        \CommentTok{\# k = 1 porque se van a hacer regresiones lineales simples}
        \CommentTok{\#y\_pred = X*beta\_1 \#Resultado predicho}
      
      \CommentTok{\# STC = suma((y\_in {-} mean(y\_in))\^{}2) \#Suma Total de Cuadrados}
    
\NormalTok{    y\_pred }\OtherTok{\textless{}{-}}\NormalTok{ matriz\_in  }\SpecialCharTok{\%*\%}\NormalTok{ beta\_1}
\NormalTok{    SRC }\OtherTok{\textless{}{-}} \FunctionTok{sum}\NormalTok{((y\_pred }\SpecialCharTok{{-}}\NormalTok{ y\_in)}\SpecialCharTok{\^{}}\DecValTok{2}\NormalTok{)}
\NormalTok{    Var\_error }\OtherTok{\textless{}{-}}\NormalTok{ SRC}\SpecialCharTok{/}\NormalTok{(}\FunctionTok{length}\NormalTok{(y\_in)}\SpecialCharTok{{-}}\DecValTok{1{-}1}\NormalTok{)}
\NormalTok{    STC }\OtherTok{\textless{}{-}} \FunctionTok{sum}\NormalTok{((y\_in }\SpecialCharTok{{-}} \FunctionTok{mean}\NormalTok{(y\_in))}\SpecialCharTok{\^{}}\DecValTok{2}\NormalTok{)}
    
\NormalTok{    eeBeta\_1 }\OtherTok{\textless{}{-}}\NormalTok{ (Var\_error}\SpecialCharTok{/}\NormalTok{STC)}\SpecialCharTok{\^{}}\FloatTok{0.5}

  \CommentTok{\#Output}
    \CommentTok{\# Error estándar de Beta\_1}
\NormalTok{   eeBeta\_1}
\NormalTok{\}}


\CommentTok{\#Estimación de variables (Prueba)}

\CommentTok{\#X \textless{}{-} matriz\_edad\_estandar}
\CommentTok{\#y\_in \textless{}{-}  matrix(outcomes\_nominales[[1]])}


\CommentTok{\#funcion\_MCO\_beta1(X,y\_in)}
\CommentTok{\#funcion\_MCO\_ee\_beta1(X,y\_in)}
\end{Highlighting}
\end{Shaded}

\hypertarget{utilizando-un-loop-apliquen-esta-funciuxf3n-a-los-diferentes-outcomes-en-las-listas-de-outcomes_nominales-guarden-los-coeficientes-estimados-y-las-desviaciones-estuxe1ndar-en-una-matriz-donde-la-primera-columna-corresponde-al-nombre-del-outcome-la-segunda-columna-al-coeficiente-estimado-ux3b21-y-la-tercera-al-error-estuxe1ndar-ux3c3ux3b2.-en-esta-matriz-cada-fila-representaruxe1-una-estimaciuxf3n.}{%
\subsubsection{2.2) Utilizando un loop, apliquen esta función a los
diferentes outcomes en las listas de outcomes\_nominales, guarden los
coeficientes estimados y las desviaciones estándar en una matriz donde
la primera columna corresponde al nombre del outcome, la segunda columna
al coeficiente estimado (β1) y la tercera al error estándar (σβ). En
esta matriz, cada fila representará una
estimación.}\label{utilizando-un-loop-apliquen-esta-funciuxf3n-a-los-diferentes-outcomes-en-las-listas-de-outcomes_nominales-guarden-los-coeficientes-estimados-y-las-desviaciones-estuxe1ndar-en-una-matriz-donde-la-primera-columna-corresponde-al-nombre-del-outcome-la-segunda-columna-al-coeficiente-estimado-ux3b21-y-la-tercera-al-error-estuxe1ndar-ux3c3ux3b2.-en-esta-matriz-cada-fila-representaruxe1-una-estimaciuxf3n.}}

\begin{Shaded}
\begin{Highlighting}[]
  \CommentTok{\#Se hace un loop en el que se concatenan los nombres de las}
  \CommentTok{\#variables, los estimadores de la regresión por MCO y sus}
  \CommentTok{\#errores estándar}
    
    \CommentTok{\#Se crea una matriz vacía con la dimensión deseada:}
    \CommentTok{\# La cantidad de filas igual a la cantidad de resultados}
    \CommentTok{\# La cantidad de columnas igual a 3: nombre, beta y error estándar}

\NormalTok{    matriz\_Estimadores }\OtherTok{\textless{}{-}} \FunctionTok{matrix}\NormalTok{(}\ConstantTok{NA}\NormalTok{,}\AttributeTok{nrow =} \FunctionTok{length}\NormalTok{(outcomes\_nominales),}
                                 \AttributeTok{ncol =} \DecValTok{3}\NormalTok{)}
    
    \ControlFlowTok{for}\NormalTok{ (i }\ControlFlowTok{in} \DecValTok{1}\SpecialCharTok{:}\FunctionTok{length}\NormalTok{(outcomes\_nominales))\{}
    
    \CommentTok{\#Input}
     
      \CommentTok{\# Funciones de estimación de betas y de errores estándar}
      \CommentTok{\# Vectores de los outcomes}
      
\NormalTok{      beta }\OtherTok{\textless{}{-}} \FunctionTok{funcion\_MCO\_beta1}\NormalTok{(matriz\_edad\_estandar,}
                                \FunctionTok{matrix}\NormalTok{(outcomes\_nominales[[i]]))}
      
\NormalTok{      error\_est }\OtherTok{\textless{}{-}} \FunctionTok{funcion\_MCO\_ee\_beta1}\NormalTok{(matriz\_edad\_estandar,}
                                \FunctionTok{matrix}\NormalTok{(outcomes\_nominales[[i]]))}
      
    \CommentTok{\#Output}
    
  
\NormalTok{      matriz\_Estimadores [i,}\DecValTok{1}\NormalTok{] }\OtherTok{\textless{}{-}}  \FunctionTok{names}\NormalTok{(outcomes\_nominales[i])}
\NormalTok{      matriz\_Estimadores [i,}\DecValTok{2}\NormalTok{] }\OtherTok{\textless{}{-}}  \FunctionTok{round}\NormalTok{(beta,}\DecValTok{3}\NormalTok{)}
\NormalTok{      matriz\_Estimadores [i,}\DecValTok{3}\NormalTok{] }\OtherTok{\textless{}{-}}  \FunctionTok{round}\NormalTok{(error\_est,}\DecValTok{3}\NormalTok{)}
     
\NormalTok{    \}}

\FunctionTok{print}\NormalTok{(matriz\_Estimadores)}
\end{Highlighting}
\end{Shaded}

\begin{verbatim}
##      [,1]          [,2]     [,3]   
## [1,] "Salario"     "-0.064" "0.144"
## [2,] "IndiceSalud" "0.052"  "0.144"
## [3,] "ExpLaboral"  "-0.045" "0.144"
\end{verbatim}

\begin{Shaded}
\begin{Highlighting}[]
\CommentTok{\# Para probar que está bien los errores estándar y los betas}
\NormalTok{x }\OtherTok{=}\NormalTok{ matriz\_edad\_estandar}
\NormalTok{y1 }\OtherTok{=} \FunctionTok{c}\NormalTok{(}\FunctionTok{matrix}\NormalTok{(outcomes\_nominales[[}\DecValTok{1}\NormalTok{]]))}
\NormalTok{y2 }\OtherTok{=} \FunctionTok{c}\NormalTok{(}\FunctionTok{matrix}\NormalTok{(outcomes\_nominales[[}\DecValTok{2}\NormalTok{]]))}
\NormalTok{y3 }\OtherTok{=} \FunctionTok{c}\NormalTok{(}\FunctionTok{matrix}\NormalTok{(outcomes\_nominales[[}\DecValTok{3}\NormalTok{]]))}

\NormalTok{datos }\OtherTok{\textless{}{-}} \FunctionTok{data.frame}\NormalTok{(x, y1, y2, y3)}

\NormalTok{modelo1 }\OtherTok{=} \FunctionTok{lm}\NormalTok{(y1 }\SpecialCharTok{\textasciitilde{}} \SpecialCharTok{{-}}\DecValTok{1} \SpecialCharTok{+}\NormalTok{ x, }\AttributeTok{data =}\NormalTok{ datos)}
\NormalTok{modelo2 }\OtherTok{=} \FunctionTok{lm}\NormalTok{(y2 }\SpecialCharTok{\textasciitilde{}} \SpecialCharTok{{-}}\DecValTok{1} \SpecialCharTok{+}\NormalTok{ x, }\AttributeTok{data =}\NormalTok{ datos)}
\NormalTok{modelo3 }\OtherTok{=} \FunctionTok{lm}\NormalTok{(y3 }\SpecialCharTok{\textasciitilde{}} \SpecialCharTok{{-}}\DecValTok{1} \SpecialCharTok{+}\NormalTok{ x, }\AttributeTok{data =}\NormalTok{ datos)}

\NormalTok{Mod1 }\OtherTok{=} \FunctionTok{summary}\NormalTok{(modelo1)}
\NormalTok{Mod2 }\OtherTok{=} \FunctionTok{summary}\NormalTok{(modelo2)}
\NormalTok{Mod3 }\OtherTok{=} \FunctionTok{summary}\NormalTok{(modelo3)}

\NormalTok{Mod1[[}\StringTok{"coefficients"}\NormalTok{]]}
\end{Highlighting}
\end{Shaded}

\begin{verbatim}
##                   Estimate Std. Error    t value  Pr(>|t|)
## xEdad estándar -0.06359524  0.1628226 -0.3905799 0.6978359
## xConstante      0.11225789  0.1611862  0.6964487 0.4895069
\end{verbatim}

\begin{Shaded}
\begin{Highlighting}[]
\NormalTok{Mod2[[}\StringTok{"coefficients"}\NormalTok{]]}
\end{Highlighting}
\end{Shaded}

\begin{verbatim}
##                  Estimate Std. Error   t value  Pr(>|t|)
## xEdad estándar 0.05220534  0.1345313 0.3880535 0.6996922
## xConstante     0.01756206  0.1331792 0.1318679 0.8956400
\end{verbatim}

\begin{Shaded}
\begin{Highlighting}[]
\NormalTok{Mod3[[}\StringTok{"coefficients"}\NormalTok{]]}
\end{Highlighting}
\end{Shaded}

\begin{verbatim}
##                   Estimate Std. Error    t value  Pr(>|t|)
## xEdad estándar -0.04515376  0.1392506 -0.3242626 0.7471481
## xConstante     -0.09558756  0.1378511 -0.6934119 0.4913930
\end{verbatim}

\begin{Shaded}
\begin{Highlighting}[]
\NormalTok{??summary}
\end{Highlighting}
\end{Shaded}

\begin{verbatim}
## starting httpd help server ... done
\end{verbatim}

\hypertarget{haciendo-uso-de-un-loop-hagan-un-print-para-que-automuxe1ticamente-y-para-cada-outcome-se-realice-la-interpretaciuxf3n-economuxe9trica-de-cada-coeficiente-de-regresiuxf3n-que-estimaron.-recuerden-tener-en-cuenta-las-distribuciones-de-las-variables-para-su-interpretaciuxf3n.}{%
\subsubsection{2.3) Haciendo uso de un loop hagan un print para que,
automáticamente y para cada outcome, se realice la interpretación
econométrica de cada coeficiente de regresión que estimaron. Recuerden
tener en cuenta las distribuciones de las variables para su
interpretación.}\label{haciendo-uso-de-un-loop-hagan-un-print-para-que-automuxe1ticamente-y-para-cada-outcome-se-realice-la-interpretaciuxf3n-economuxe9trica-de-cada-coeficiente-de-regresiuxf3n-que-estimaron.-recuerden-tener-en-cuenta-las-distribuciones-de-las-variables-para-su-interpretaciuxf3n.}}

\begin{Shaded}
\begin{Highlighting}[]
\ControlFlowTok{for}\NormalTok{ (i }\ControlFlowTok{in} \DecValTok{1}\SpecialCharTok{:}\FunctionTok{length}\NormalTok{(outcomes\_nominales))\{}

    \FunctionTok{print}\NormalTok{(}\FunctionTok{paste}\NormalTok{(}\StringTok{"Por cada incremento de una desviación estándar de la edad, la variable"}\NormalTok{,matriz\_Estimadores[i,}\DecValTok{1}\NormalTok{],}\StringTok{"cambia en"}\NormalTok{,matriz\_Estimadores[i,}\DecValTok{2}\NormalTok{],}\StringTok{"desviaciones estándar"}\NormalTok{))}
\NormalTok{\}}
\end{Highlighting}
\end{Shaded}

\begin{verbatim}
## [1] "Por cada incremento de una desviación estándar de la edad, la variable Salario cambia en -0.064 desviaciones estándar"
## [1] "Por cada incremento de una desviación estándar de la edad, la variable IndiceSalud cambia en 0.052 desviaciones estándar"
## [1] "Por cada incremento de una desviación estándar de la edad, la variable ExpLaboral cambia en -0.045 desviaciones estándar"
\end{verbatim}

\begin{Shaded}
\begin{Highlighting}[]
\CommentTok{\#for(interpretación in 1:length(matrizest[1,]))\{}
  
 \CommentTok{\# print()}
  
\CommentTok{\#\}}
\end{Highlighting}
\end{Shaded}

\hypertarget{creen-una-funciuxf3n-que-calcule-el-msey_ix_i-ux3b2_0ux3b2_1.-es-decir-que-tenga-como-input-un-vector-de-yi-un-vectorlista-de-xi-y-los-paruxe1metros-ux3b20-y-ux3b21.-como-output-debe-proveer-un-escalar-correspondiente-al-mse-de-esa-combinaciuxf3n-de-inputs.}{%
\subsubsection{2.4) Creen una función que calcule el MSE(y\_i,x\_(i,)
β\_0,β\_1). Es decir, que tenga como input un vector de yi, un
vector/lista de xi y los parámetros β0 y β1. Como output debe proveer un
escalar correspondiente al MSE de esa combinación de
inputs.}\label{creen-una-funciuxf3n-que-calcule-el-msey_ix_i-ux3b2_0ux3b2_1.-es-decir-que-tenga-como-input-un-vector-de-yi-un-vectorlista-de-xi-y-los-paruxe1metros-ux3b20-y-ux3b21.-como-output-debe-proveer-un-escalar-correspondiente-al-mse-de-esa-combinaciuxf3n-de-inputs.}}

\begin{Shaded}
\begin{Highlighting}[]
\CommentTok{\#Se crea una función para determinar el MSE}

\NormalTok{funcion\_MSE }\OtherTok{\textless{}{-}} \ControlFlowTok{function}\NormalTok{(y\_in,X\_in,beta0,beta1)\{}
  
  \CommentTok{\#Input}
    \CommentTok{\#y\_in: Un vector del resultado}
    \CommentTok{\#X\_in: Una matriz de los valores independientes}
    \CommentTok{\#beta0: Valor de la constante de la regresión lineal simple}
    \CommentTok{\#beta1: Estimador de la regresión para la variable independiente}
  
  \CommentTok{\#Output}
    \CommentTok{\#MSE: Error al cuadrado promedio}
  
\NormalTok{  MSE }\OtherTok{\textless{}{-}} \FunctionTok{sum}\NormalTok{((y\_in }\SpecialCharTok{{-}}\NormalTok{ beta0 }\SpecialCharTok{{-}}\NormalTok{ beta1}\SpecialCharTok{*}\NormalTok{X\_in)}\SpecialCharTok{\^{}}\DecValTok{2}\NormalTok{)}\SpecialCharTok{/}\FunctionTok{length}\NormalTok{(y\_in)}
  
  
  
\NormalTok{\}}

\FunctionTok{funcion\_MSE}\NormalTok{(}\FunctionTok{matrix}\NormalTok{(outcomes\_nominales[[}\DecValTok{1}\NormalTok{]]),}
            \FunctionTok{matrix}\NormalTok{(matriz\_edad\_estandar[,}\DecValTok{1}\NormalTok{]),}\DecValTok{1}\NormalTok{,}\DecValTok{1}\NormalTok{)}
\end{Highlighting}
\end{Shaded}

\hypertarget{utilizando-un-ciclo-while-generen-una-funciuxf3n-que-retorne-el-coeficiente-ux3b2min1-que-minimiza-el-error-de-ajuste-de-los-datos-de-forma-numuxe9rica.}{%
\subsubsection{2.5) Utilizando un ciclo while generen una función que
retorne el coeficiente βmin\^{}1 que minimiza el error de ajuste de los
datos de forma
numérica.}\label{utilizando-un-ciclo-while-generen-una-funciuxf3n-que-retorne-el-coeficiente-ux3b2min1-que-minimiza-el-error-de-ajuste-de-los-datos-de-forma-numuxe9rica.}}

\begin{Shaded}
\begin{Highlighting}[]
\NormalTok{funcion\_Min\_MSE }\OtherTok{\textless{}{-}} \ControlFlowTok{function}\NormalTok{(y\_in,X\_in,beta) \{}

\CommentTok{\#Valores iniciales}

\NormalTok{i }\OtherTok{\textless{}{-}} \SpecialCharTok{{-}}\DecValTok{1} \CommentTok{\#Diferencia entre MSE}
\NormalTok{beta\_1 }\OtherTok{\textless{}{-}} \SpecialCharTok{{-}}\DecValTok{5} \CommentTok{\#Valor inicial de beta}
\NormalTok{paso\_iteracion }\OtherTok{\textless{}{-}} \FloatTok{0.001} \CommentTok{\#El incremento de beta\_1 por iteración}
\NormalTok{MSE\_t\_1 }\OtherTok{\textless{}{-}} \FunctionTok{funcion\_MSE}\NormalTok{(y\_in,X\_in,}\DecValTok{0}\NormalTok{,(beta\_1}\SpecialCharTok{{-}}\NormalTok{paso\_iteracion)) }\CommentTok{\#Valor inicial de MSE, como es la inicial decidimos usar beta con un paso menos}

\ControlFlowTok{while}\NormalTok{(i}\SpecialCharTok{\textless{}}\DecValTok{0}\NormalTok{)\{}
  
  \CommentTok{\#Se utiliza la función MSE del punto anterior}
  
  \FunctionTok{invisible}\NormalTok{(MSE }\OtherTok{\textless{}{-}} \FunctionTok{funcion\_MSE}\NormalTok{(y\_in,X\_in,}\DecValTok{0}\NormalTok{,beta\_1))}
  
  \CommentTok{\#Se calcula la diferencia entre el MSE actual y el de la iteración}
  \CommentTok{\#anterior}
\NormalTok{  i }\OtherTok{\textless{}{-}}\NormalTok{ MSE }\SpecialCharTok{{-}}\NormalTok{ MSE\_t\_1}
  
  \CommentTok{\#Se actualiza el valor del MSE para pasar a la siguiente iteración}
  
\NormalTok{  MSE\_t\_1 }\OtherTok{\textless{}{-}}\NormalTok{ MSE}
  
  \CommentTok{\#Se incrementa el valor de beta en un valor constante dado por}
  \CommentTok{\#el valor del paso de la iteración}
\NormalTok{  beta\_1 }\OtherTok{\textless{}{-}}\NormalTok{ beta\_1 }\SpecialCharTok{+}\NormalTok{ paso\_iteracion}
  
\NormalTok{  \}}
\NormalTok{beta\_1}
\NormalTok{\}}
\end{Highlighting}
\end{Shaded}

\hypertarget{apliquen-las-funciones-desarrolladas-en-los-puntos-anteriores-para-encontrar-el-ux3b2min_1-que-minimiza-el-error-cuadruxe1tico-medio-por-muxe9todos-numuxe9ricos-suponiendo-que-ux3b2_00-para-los-outcomes_nominales.-reporten-sus-resultados-en-una-matriz-con-dos-columnas-nombre-del-outcome-y-ux3b2min_1.-esta-matriz-debe-tener-3-filas-una-por-cada-outcome}{%
\subsubsection{2.6) Apliquen las funciones desarrolladas en los puntos
anteriores para encontrar el 〖β\^{}min〗\_1 que minimiza el error
cuadrático medio por métodos numéricos, suponiendo que β\_0=0, para los
outcomes\_nominales. Reporten sus resultados en una matriz con dos
columnas: Nombre del outcome y 〖β\^{}min〗\_1. Esta matriz debe tener 3
filas, una por cada
outcome}\label{apliquen-las-funciones-desarrolladas-en-los-puntos-anteriores-para-encontrar-el-ux3b2min_1-que-minimiza-el-error-cuadruxe1tico-medio-por-muxe9todos-numuxe9ricos-suponiendo-que-ux3b2_00-para-los-outcomes_nominales.-reporten-sus-resultados-en-una-matriz-con-dos-columnas-nombre-del-outcome-y-ux3b2min_1.-esta-matriz-debe-tener-3-filas-una-por-cada-outcome}}

\begin{Shaded}
\begin{Highlighting}[]
\NormalTok{matriz\_Est\_numerico }\OtherTok{\textless{}{-}} \FunctionTok{matrix}\NormalTok{(}\ConstantTok{NA}\NormalTok{,}\AttributeTok{nrow =} \FunctionTok{length}\NormalTok{(outcomes\_nominales),}
                                 \AttributeTok{ncol =} \DecValTok{2}\NormalTok{)}

\ControlFlowTok{for}\NormalTok{ (i }\ControlFlowTok{in} \DecValTok{1}\SpecialCharTok{:}\FunctionTok{length}\NormalTok{(outcomes\_nominales))\{}
  
  \CommentTok{\#input}
    \CommentTok{\#vector de y\_in}
    \CommentTok{\#matriz de X}
    \CommentTok{\#beta\_0 = 0}
  
  \CommentTok{\#Se hace un loop para que abarque todos los outcomes}
  
\NormalTok{  matriz\_Est\_numerico [i,}\DecValTok{1}\NormalTok{] }\OtherTok{\textless{}{-}} \FunctionTok{names}\NormalTok{(outcomes\_nominales[i])}
\NormalTok{  matriz\_Est\_numerico [i,}\DecValTok{2}\NormalTok{] }\OtherTok{\textless{}{-}} \FunctionTok{round}\NormalTok{(}\FunctionTok{funcion\_Min\_MSE}\NormalTok{(}
                                \FunctionTok{matrix}\NormalTok{(outcomes\_nominales[[i]]),}
                                \FunctionTok{matrix}\NormalTok{(matriz\_edad\_estandar[,}\DecValTok{1}\NormalTok{]),}\DecValTok{0}\NormalTok{),}\DecValTok{3}\NormalTok{)}
\NormalTok{\}}
\NormalTok{matriz\_Est\_numerico}
\end{Highlighting}
\end{Shaded}

\begin{verbatim}
##      [,1]          [,2]    
## [1,] "Salario"     "-0.062"
## [2,] "IndiceSalud" "0.054" 
## [3,] "ExpLaboral"  "-0.043"
\end{verbatim}

\end{document}
